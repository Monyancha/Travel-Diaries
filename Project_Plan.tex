\documentclass[]{article}
\usepackage{graphicx}
\usepackage{xcolor}
\usepackage{enumerate}
\usepackage{hyperref}
\hypersetup{%
  colorlinks = true,
  linkcolor  = black
}


% Title Page
\title{Software Engineering (CS301)\\ Document Name(Project Plan Version -: 1.0)\\Travel Diaries}
\author{Group 5}

\begin{document}
\maketitle


\begin{center}
\textbf{Project Members}\\
\vspace*{.6cm}
\begin{tabular}{|c|c|}
\hline
\textbf{ID} & \textbf{Name}\\
\hline
\hline
201452004 & Nilesh Chaturvedi\\
\hline
201452005 & Jitendra Singh\\
\hline
201452012 & Durga Vijaya Lakshmi\\
\hline
201452036 & Pedapalli Akhil\\
\hline
20152040 & B. Indu\\
\hline
201452044 & Dileep Krishna\\
\hline
201452050 & Shreya Singh\\
\hline
201452056 & Ravi Kumar Patel\\
\hline
201452057 & G. Raju Koushik\\
\hline
\end{tabular}

\vspace*{1cm}

\begin{tabular}{|c|c|}
\hline
\textbf{Authored By} & \textbf{Raju Kaushik }\\
\textbf{Reviewed By} & \textbf{Ravi Kumar Patel}\\
\hline
\end{tabular}
\end{center}

\newpage
\tableofcontents
\newpage


\section{Roles and Responsibilities.}
\subsection{Purpose -:}
The purpose of this document to mention the roles of team member for the project and also
defines the timeline of work products or milestones of each and every phase. It works as the
reference for the achievement of work products timely. It also defines about our assumption and
constraints of our team, Risk of our project and how we manage the quality of the project
\subsection{Project Overview -:}
The Offline typeracer is an offline desktop application that allows a user to learn typing,
and to compete within a group of 4 or 5 using wireless LAN. In this application, user can grow
his typing skill by practicing as a single player or compete with his desirable friend in the group
of 4 or 5 people. We aim to provide it in such a way that user can also get some extra knowledge
during typing by providing him some educational topic on which he/she wants to type and also
can have fun with his friend during match time

\subsection{Roles -:}

\begin{center}
\textbf{}\\
\vspace*{.6cm}
\begin{tabular}{|c|c|}
\hline
\textbf{Student} & \textbf{Roles}\\
\hline
\hline
Nilesh Chaturvedi &\begin{tabular}{@{}c@{}} 1. Android Team \\ 2. Low Level Design Team \\3. Deployment Team  \end{tabular}\\
\hline
Jitendra Singh & \begin{tabular}{@{}c@{}} 1. Documentation team  \\ 2. Project Proposal \\3. Testing Team \end{tabular}\\
\hline
Durga Vijaya Lakshmi & \begin{tabular}{@{}c@{}} 1. Feasibility study \\ 2. Frontend Team \\ 3. High Level Design Team\end{tabular}\\
\hline
Pedapalli Akhil & \begin{tabular}{@{}c@{}} 1. Documentation team \\ 2. Project Proposal \\3. High Level Design Team \end{tabular}\\
\hline
B. Indu & \begin{tabular}{@{}c@{}} 1. Documentation team \\ 2. Low Level Design Team \\3. Deployment Team\end{tabular}\\
\hline
Dileep Krishna & \begin{tabular}{@{}c@{}} 1. Feasibility study \\ 2. Frontend Team \\3. Testing Team \end{tabular}\\
\hline
Shreya Singh & \begin{tabular}{@{}c@{}} 1. Documentation team \\ 2. Frontend Team \\3. Testing Team\end{tabular}\\
\hline
Ravi Kumar Patel & \begin{tabular}{@{}c@{}} 1. Android Team \\ 2. Project Proposal \\3. Backend Team \end{tabular}\\
\hline
G. Raju Koushik & \begin{tabular}{@{}c@{}} 1. Android Team \\ 2. Deployment Team \\3. Backend Team \\4. High Level Design Team\end{tabular}\\
\hline
\end{tabular}



\end{center}

\newpage
\tableofcontents

\subsection{Timeline -:}
There is expected timeline of the project which determine the starting and end date of
each and every activity as concern to our project.


\begin{center}
\textbf{}\\
\vspace*{.6cm}
\begin{tabular}{|c|c|c|c|}
\hline
\textbf{Phases} & \textbf{Start Date} &\textbf{End Date} & \textbf{Milestone} \\
\hline
\hline

Feasibility & 27-08-2016  &12-09-2016 & \begin{tabular}{@{}c@{}} 1. Feasibility study \\ 2. Project Proposal \end{tabular}\\
\hline

Requirement & 10-09-2016  & 29-09-2016 &  \begin{tabular}{@{}c@{}} 1. System Requirement \\Specification
 \\ 2. Gannt chart \\ 3. Traceability Matrix \\ 4. Sdlc Model \\ 5. Cost Estimation  \\ 6. Project Plan
 \end{tabular}\\
\hline
Design & 27-09-2016  & 08-10-2016 & \begin{tabular}{@{}c@{}} 1. System test Plan \\2. Draft user manual \\3. ERD \\4. Design
Documents \end{tabular}\\
\hline
Coding and unit testing & 08-10-2016 & 12-11-2016 &\begin{tabular}{@{}c@{}} 1. Individually tested \\ modules \\2. Quality Assurance
 \\control
\end{tabular}\\
\hline
Testing & 12-11-2016  & 14-11-2016 &\begin{tabular}{@{}c@{}} 1. Analysis Report \\ 2. Complete Integrated \\product \\ 3. Test Report \\4. Tested System
\end{tabular}\\
\hline 
Deployment and Maintenance & 14-11-2016  & 18-11-2016 &\begin{tabular}{@{}c@{}} 1. Deployment issues

\end{tabular}\\
\hline 

\end{tabular}



\end{center}

\subsection{Monitoring and scheduling -:}
Monitoring and scheduling of the project will be managed by estimating the efforts and
time required for each and every phase, with the help of this a schedule will be prepared.
This schedule will be used to monitor the progress of the project. And also regularly
meetings(once or twice(if needed) in a week) will help us to keep track of the progress of
the project. For each and every activity, team will be divided into sub-groups with the
assigned tasks and after completion, will be discussed among all. Also we are using some
tools to schedule our project and track its progress, i.e. Gantt chart, Activity graph etc.

\subsection{Assumption and constraints -:}
\begin{itemize}


\item Team will work almost eight hours(per person) in a week.
\item  Team will not work if members have other stuffs related to academic.
\item It may happen that we will not be able to include all the functionalities based on
requirement analysis in our first version of project.
\item As concerned to multiplayer feature, Presently we are trying to connect 3-4 users through
hotspot.
\item We have mid-sem exam from 19/09/2016 to 24/09/2016, so we will not be able to do any
work during exams.
\item We have college cultural fest on 21/10/2016, so we will not be able to do any work
during that.
\item There is Diwali break from 30/10/2016 to 04/11/2016, so we will not do any work during
the vacations.
\end{itemize}
\subsection{Quality Control -:}
Quality control will be managed by Error tracking technique which allows comparison of
current work to past project, provides a quantitative indication of the quality of the work being
conducted. And also testing will play a lead role in quality control of the final product.
\subsection{Risk Management -:}
\begin{itemize}
\item Due to lack of enough technical skills as concerned to our project, our final product may
affected or deployment of product may vary from the time to deploy.
\item Major problem is to manage multiple players at a time and to handle the audio map
feature. The probability of occurring this problem is significance.
\item Quality can be affected due to inadequate knowledge about technique, programming
language and tools etc. 
\end{itemize}
% Install Python
%\begin{itemize}
% 
%
%\item Contents
% 
% \item Install Apache and mod_wsgi
% \item Get your database running.
% \item Remove any old versions of Django.
% \item Install the Django code. Installing an official release with pip. Installing a distribution-specific package. Installing     \item the development version.
%
%\end{itemize}


\end{document}
