\documentclass[]{article}
\usepackage{graphicx}
\usepackage{xcolor}
\usepackage{enumerate}
\usepackage{hyperref}
\hypersetup{%
  colorlinks = true,
  linkcolor  = black
}


% Title Page
\title{Software Engineering (CS301)\\ Document Name(Version)\\Travel Diaries}
\author{Group 5}

\begin{document}
\maketitle


\begin{center}
\textbf{Project Members}\\
\vspace*{.6cm}
\begin{tabular}{|c|c|}
\hline
\textbf{ID} & \textbf{Name}\\
\hline
\hline
201452004 & Nilesh Chaturvedi\\
\hline
201452005 & Jitendra Singh\\
\hline
201452012 & Durga Vijaya Lakshmi\\
\hline
201452036 & Pedapalli Akhil\\
\hline
20152040 & B. Indu\\
\hline
201452044 & Dileep Krishna\\
\hline
201452050 & Shreya Singh\\
\hline
201452056 & Ravi Patel\\
\hline
201452057 & G. Raju Koushik\\
\hline
\end{tabular}

\vspace*{1cm}

\begin{tabular}{|c|c|}
\hline
\textbf{Authored By} & \textbf{Name}\\
\textbf{Reviewed By} & \textbf{Name}\\
\hline
\end{tabular}
\end{center}

\newpage
\tableofcontents
\newpage


\section{Introduction}
The application shall allow users to maintain their own login account and would allow the user to set
their privacy levels according to their convenience. As the user logs in to their respective accounts, the
application would not only provide the users with the appropriate posts of travel diaries but it will also
enable them to upload their experiences in the specific travel diary they intend to. Moreover, they can
share their experiences along with their geo-tagged photos on the platform and would help out people
in deciding a destination.
\section{System Environment}
\begin{itemize}
\item Development: Android SDK
%\item Diagrams: Wireframe, Dia
\item Database Management: PostgreSQL 
\item Server: Django
%\item Discussion: Whatsapp Group
\end{itemize}
\section{Setting Up Android Studio and server }
The website executes on a web server running Django . So before you start using website you need following programs(softwares) installed on computer/system :
\begin{itemize}
\item Android Stduio
\item Sdk
\item PostgreSQL database server
\item pgAdminIII
\item Django server
\end{itemize}

Based on the platform, like if you are using \textbf{Linux} you have to follow the given procedures:\\
\subsection{Installing Android studio}
Go to \href{https://developer.android.com/studio/index.html}{Android Developer}. home page and download according to your environment and follow the instruction .


\subsection{Install Django}
\begin{enumerate}
   \item First Install Python
   
   \begin{enumerate}
     \item 
     \item Second level item
     \begin{enumerate}
       \item Third level item
       \item Third level item
       \begin{enumerate}
         \item Fourth level item
         \item Fourth level item
       \end{enumerate}
     \end{enumerate}
   \end{enumerate}
 \end{enumerate}




% Install Python
%\begin{itemize}
% 
%
%\item Contents
% 
% \item Install Apache and mod_wsgi
% \item Get your database running.
% \item Remove any old versions of Django.
% \item Install the Django code. Installing an official release with pip. Installing a distribution-specific package. Installing     \item the development version.
%
%\end{itemize}




$\gg$For PostgreSQL we have to install PHP module for PostgreSQL, for that type:


\subsection{Install PostgreSQL and Setup Database Server}
$\gg$Open Terminal and type the given commands:

$\gg$Now type following commands to install Administration tool \textbf{pgAdminIII}

\subsubsection{Setting up pgAdminIII}
$\gg$Open pgAdminIII tool

$\gg$Click top-left icon

$\gg$A window pops-out

$\gg$Fill the entries accordingly

$\gg$Once the server is setup then restore the backup file which is included in this software.
$\gg$Backup file will create the database and make all the entries which are required.
\subsubsection{Running the server}
\begin{itemize}
\item We will provide \texttt{backup file} for the database.
\item In pgAdmin, click on the web server you just created.
\item Right click on the Database heading, select New Database and name it IIITV.

\item If on the different pc (Note - both pc must be connected to the same network), then find the IP address of the pc in which database is created using \texttt{ifconfig} command. After that, in the second pc, open \texttt{db$\_$conn.djago} file and in place of \texttt{localhost}, type the IP of the first pc. Follow the above steps.
\end{itemize}

\end{document}
